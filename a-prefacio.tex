% !TEX root = z-main.tex

Este trabajo representa la culminación de un esfuerzo colectivo e individual que no habría sido posible sin el apoyo y la dedicación de personas excepcionales que me acompañaron en este camino.

Agradezco profundamente a mis compañeros de equipo, quienes con su compromiso y talento hicieron posible que este proyecto cobrara vida. A Carlos López, por su trabajo en la integración de modelos en la aplicación, esencial para dar vida a las reconstrucciones digitales. A Brian Carrillo, cuyo trabajo en la implementación del sistema de geolocalización fue fundamental para anclar la experiencia digital al espacio físico del parque. A Marco Ramírez, quien dedicó incontables horas al modelado tridimensional de las estructuras arqueológicas, logrando reconstrucciones que honran el patrimonio cultural guatemalteco. A Luz Coronado, cuya visión estratégica en la gestión y el análisis de recursos aseguró la viabilidad del proyecto. A Claudia Velásquez, cuya investigación arqueológica aportó el rigor histórico necesario para que el contenido fuera fiel a la realidad de Kaminaljuyú. Trabajar junto a ustedes fue un privilegio y una experiencia de aprendizaje invaluable.

A mi asesor de tesis, M. Ed. Dennis Moritz Aldana Moscoso, por su guía constante, su paciencia y sus valiosas retroalimentaciones que enriquecieron cada etapa de esta investigación. Sus enseñanzas trascendieron lo académico y me formaron como profesional.

A la Mgtr. Inga. Dulce Chacón, por acompañarnos durante este proceso con su experiencia, su apertura al diálogo y su compromiso con la excelencia académica. Su presencia fue un pilar fundamental en el desarrollo de este trabajo.

A mi padre, Raúl Morales, por haberme permitido llegar hasta este momento. Tu esfuerzo y sacrificio son la base sobre la que construí este logro. A mi madre, Amelia González, por haber sido un apoyo incondicional durante toda mi vida, por creer en mí incluso cuando yo dudaba, y por enseñarme el valor de la perseverancia. A mis hermanos, David Morales y Helen Morales, por ser ejemplos a seguir, grandes compañeros de vida y una fuente constante de motivación. Los admiro profundamente.

Finalmente, agradezco al Parque Arqueológico Kaminaljuyú por ser la inspiración de este proyecto. Espero que esta aplicación contribuya, aunque sea modestamente, a preservar y difundir el invaluable patrimonio cultural que Guatemala resguarda.
