% !TEX root = z-main.tex

El Parque Arqueológico Kaminaljuyú, ubicado en el corazón de la Ciudad de Guatemala, constituye uno de los sitios más representativos de la civilización maya del período preclásico. A pesar de su valor histórico, arqueológico y cultural, el parque enfrentaba desafíos importantes en cuanto a su interpretación y difusión al público general. La escasa señalización, la falta de contextualización accesible para diversos perfiles de visitantes y la ausencia de herramientas didácticas modernas provocaban una experiencia que resultaba fragmentada y limitada, especialmente para quienes no poseían conocimientos previos sobre la cultura maya.

En este contexto, se identificó que el diseño de experiencias centradas en el usuario (UX) y de interfaces de usuario (UI) que fueran intuitivas, visualmente atractivas y funcionales adquiría un rol esencial para transformar la forma en que se interactúa con el patrimonio arqueológico. La incorporación de tecnologías emergentes, como la Realidad Aumentada (RA), abrió nuevas posibilidades para crear entornos inmersivos que facilitaran la comprensión del entorno físico y su valor histórico. A través de elementos visuales superpuestos al mundo real, la RA permitió reconstruir virtualmente estructuras, narrar historias y ofrecer datos clave en tiempo real, enriqueciendo la exploración del parque de manera significativa.

Este proyecto desarrolló una aplicación de recorrido con tecnología de Realidad Aumentada, diseñada específicamente para Kaminaljuyú, con el objetivo de mejorar la experiencia de los visitantes. El enfoque combinó principios de diseño centrado en el usuario, accesibilidad y usabilidad, buscando fomentar la interacción activa, el interés por el pasado precolombino y el aprecio por el patrimonio cultural del país. Además de facilitar una navegación más informada y atractiva por el parque, la solución logró democratizar el acceso al conocimiento arqueológico, brindando a los usuarios una experiencia educativa y memorable, alineada con las necesidades de los públicos contemporáneos.