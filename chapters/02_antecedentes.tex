% !TEX root = ../main.tex

La integración de tecnologías inmersivas en la preservación y difusión del patrimonio cultural ha experimentado un crecimiento significativo en los últimos años, particularmente con el desarrollo de aplicaciones de realidad aumentada dirigidas a sitios arqueológicos y museos. Sin embargo, el éxito de estas implementaciones no depende únicamente de la sofisticación tecnológica, sino de manera fundamental del diseño de experiencias de usuario intuitivas, accesibles y culturalmente pertinentes.

En este contexto, el estudio de interfaces tangibles para aplicaciones de realidad virtual en contextos de patrimonio cultural ha demostrado su relevancia para mejorar la experiencia educativa y la inmersión del usuario. Una investigación publicada en el Journal on Computing and Cultural Heritage examinó el impacto de interfaces de usuario tangibles en aplicaciones de VR para patrimonio cultural, evaluando específicamente la usabilidad del sistema, la experiencia de usuario y el nivel de inmersión alcanzado \cite{tangibleinterfaces2023}. Los resultados confirmaron que las interfaces tangibles mejoran significativamente la experiencia del usuario y aumentan el disfrute del aprendizaje sobre patrimonio cultural, lo que sugiere que la interacción física puede complementar exitosamente las experiencias inmersivas digitales. Esta investigación proporcionó lineamientos para el diseño de interfaces que equilibran innovación tecnológica con facilidad de uso, un principio fundamental para cualquier aplicación orientada al público general.

Complementando estos hallazgos, el diseño centrado en el usuario ha emergido como un enfoque esencial para garantizar el éxito de aplicaciones de realidad aumentada, particularmente en contextos culturales donde la diversidad de perfiles de visitantes exige interfaces adaptables y accesibles. El estándar internacional ISO 9241-210 define el diseño centrado en el humano como un enfoque que involucra a los usuarios a lo largo del proceso de desarrollo, involucrando investigación de usuarios, caracterización de problemas, ideación, prototipado y pruebas iterativas \cite{iso2019ergonomics}. En el contexto de aplicaciones móviles para patrimonio cultural, los factores humanos resultan fundamentales para la efectividad narrativa, ya que las aplicaciones deben evitar distraer la atención del usuario del contenido para fomentar una buena concentración e inmersión \cite{ramtohul2023user}. Ramtohul y Khedo (2023) enfatizan que los propietarios de patrimonio cultural deben repensar y desarrollar enfoques centrados en el visitante para crear experiencias de usuario más atractivas, reconociendo que, a pesar del uso generalizado de RA en patrimonio cultural, no existen suficientes estudios sobre la experiencia de usuario, los resultados de aprendizaje y la forma en que los usuarios observan e interactúan con el contenido virtual.

Esta necesidad de profundizar en el diseño de interacciones se ve reflejada en una revisión sistemática de 64 publicaciones entre 2016 y 2023 que identificó y analizó patrones de diseño de interacción UX para aplicaciones de RA móvil en patrimonio cultural, estableciendo las bases para la reflexión sobre enfoques actuales entre diseñadores UX y profesionales del patrimonio cultural \cite{mobileARpatterns2024}. Este estudio reveló que, si bien la mayor parte de la literatura se centra en aspectos tecnológicos, ha habido poca exploración de las capacidades expresivas de la RA para interacciones significativas. Los patrones identificados sugieren que el diseño de interacciones debe considerar no solo la funcionalidad técnica, sino también la capacidad de la tecnología para transmitir narrativas culturales de manera efectiva y emocionalmente resonante. Estas guías de diseño se organizan en categorías que incluyen diseño ambiental, diseño de interacción y señales visuales, proporcionando un marco metodológico para evaluar desafíos y oportunidades en el diseño de experiencias AR para patrimonio.

La evaluación de la usabilidad, la experiencia de usuario y la carga mental de trabajo en aplicaciones de RA móvil para narrativa digital en patrimonio cultural ha demostrado ser fundamental para comprender cómo los usuarios procesan información en entornos aumentados. Un estudio específico sobre estos aspectos concluyó que las aplicaciones de RA generan una experiencia de usuario positiva y una evidente ganancia de aprendizaje, siendo consideradas fáciles de usar, lo que resalta su potencial para ser ampliamente adoptadas en edificios con valor arquitectónico \cite{usabilityARheritage2023}. Esta investigación enfatizó que la RA mejora la experiencia del usuario y aumenta el disfrute del aprendizaje sobre patrimonio cultural, confirmando que la tecnología puede crear las conexiones faltantes en entornos patrimoniales estáticos. Sin embargo, el estudio también advirtió sobre la importancia de mantener una carga cognitiva manejable para evitar la saturación del usuario, especialmente en contextos donde se presenta información histórica compleja.

En términos de implementaciones prácticas, el proyecto HeritageSite AR, publicado en el Journal on Computing and Cultural Heritage, representa un caso de estudio significativo en el diseño y evaluación de juegos de exploración con realidad aumentada móvil para sitios patrimoniales chinos \cite{heritagesiteAR2024}. El desarrollo de esta aplicación involucró entrevistas semiestructuradas con expertos del dominio y la administración de encuestas para identificar requisitos de usuarios y objetivos de diseño. Los resultados demostraron que los juegos de exploración AR móvil pueden mejorar la narrativa inmersiva y enriquecer las experiencias culturales, proporcionando un modelo de desarrollo participativo que integra las perspectivas de expertos en patrimonio con las necesidades de los usuarios finales. Este enfoque metodológico destaca la importancia de la investigación formativa y la evaluación iterativa en el diseño de experiencias AR culturales efectivas.

Profundizando en los marcos conceptuales, una investigación publicada en marzo de 2024 construyó modelos detallados de experiencia de usuario utilizando codificación teórica y propuso estrategias de diseño específicas para exposiciones de patrimonio cultural basadas en realidad virtual \cite{uxmodelVR2024}. Este estudio reconoció que el diseño de experiencia de usuario ha encontrado nuevas oportunidades debido a las diferencias en la interacción humano-computadora entre interfaces VR y convencionales. Las estrategias de diseño propuestas abordan aspectos como la navegación intuitiva, la accesibilidad cognitiva, la inmersión sensorial y la coherencia narrativa, elementos que resultan igualmente aplicables al diseño de experiencias AR. El modelo de UX desarrollado ofrece un marco conceptual para evaluar y optimizar la experiencia del usuario en entornos inmersivos patrimoniales, enfatizando la necesidad de equilibrar innovación tecnológica con facilidad de uso.

En el contexto guatemalteco, el proyecto KAN ha representado un avance significativo en la aplicación de tecnologías inmersivas al patrimonio arqueológico. Esta iniciativa surgió como una propuesta interdisciplinaria entre el sector privado, instituciones culturales y arqueólogos del país, con el objetivo de ofrecer una experiencia histórica enriquecida a los visitantes del Parque Arqueológico Kaminaljuyú \cite{GarridoFlores2021}. La aplicación móvil desarrollada permite superponer reconstrucciones 3D de estructuras mayas sobre las ruinas actuales mediante tecnologías de realidad aumentada, como ARKit y ARCore, junto con modelado en Blender. El proyecto ha sido reconocido por su valor cultural y educativo, recibiendo el primer lugar en la Creative Business Cup Guatemala 2021. Su éxito como piloto abre la posibilidad de implementación en otros parques arqueológicos del país, promoviendo el uso de RA como herramienta para el turismo y la educación patrimonial. Sin embargo, si bien el proyecto KAN demostró la viabilidad técnica de la RA en Kaminaljuyú, no se enfocó específicamente en el diseño de experiencia de usuario ni en la evaluación sistemática de usabilidad y accesibilidad, aspectos que resultan fundamentales para garantizar una adopción amplia y efectiva de la tecnología.

La tendencia hacia implementaciones más sofisticadas de RA se evidencia en casos recientes a nivel internacional. Durante 2024, diversos museos han implementado experiencias de realidad aumentada que ilustran diferentes enfoques de diseño de interfaz y experiencia de usuario. El Museo de Historia Natural de Londres lanzó en octubre de 2024 "Visions of Nature", una experiencia inmersiva de realidad mixta que transporta a los visitantes al año 2125 para ilustrar el impacto de las acciones humanas en el planeta \cite{museumAR2024}. Por su parte, el museo de Young en San Francisco integró en enero de 2024 una instalación interactiva de prueba virtual en colaboración con Snap Inc., permitiendo a los visitantes vestirse virtualmente con conjuntos de alta costura de diseñadores como Yves Saint Laurent y Valentino. Estos casos demuestran que las implementaciones exitosas de AR en museos requieren no solo tecnología robusta, sino también interfaces intuitivas que permitan a usuarios diversos interactuar con el contenido digital sin necesidad de conocimientos técnicos previos. El diseño de estas experiencias priorizó la accesibilidad, la navegación clara y la integración fluida entre el mundo físico y el contenido digital, principios que resultan transferibles al diseño de aplicaciones AR para sitios arqueológicos al aire libre.

Los antecedentes revisados revelan que, si bien la tecnología de realidad aumentada ha sido aplicada exitosamente en diversos contextos de patrimonio cultural, persiste una brecha significativa en el diseño sistemático de experiencias de usuario para aplicaciones AR en sitios arqueológicos latinoamericanos, particularmente en el contexto maya guatemalteco. Aunque el proyecto KAN estableció un precedente valioso en Kaminaljuyú, la ausencia de un enfoque metodológico centrado en el diseño de UX/UI representa una oportunidad para desarrollar una aplicación que no solo sea técnicamente funcional, sino que también ofrezca una experiencia coherente, intuitiva y culturalmente significativa. La literatura revisada subraya la importancia de aplicar principios de diseño centrado en el usuario, realizar evaluaciones iterativas de usabilidad y desarrollar interfaces que equilibren innovación tecnológica con accesibilidad, aspectos que constituyen el núcleo del presente trabajo de graduación.
