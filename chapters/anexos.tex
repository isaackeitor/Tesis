% !TEX root = ../main.tex

\section{Anexo A: Fotografías de las pruebas de usabilidad}

Las siguientes fotografías documentan las sesiones de pruebas de usabilidad realizadas con estudiantes universitarios. Durante estas sesiones se evaluó la facilidad de uso, navegación y experiencia general de la aplicación de realidad aumentada para el Parque Arqueológico Kaminaljuyú.

\vspace{0.5cm}

\begin{figure}[h!]
    \centering
    \includegraphics[width=0.55\textwidth]{imagenes/pruebasconusuarios.jpeg}
    \caption{Sesión de prueba de usabilidad 1.}
    \label{fig:anexo-prueba1}
\end{figure}

\vspace{0.5cm}

\begin{figure}[h!]
    \centering
    \begin{subfigure}[b]{0.32\textwidth}
        \centering
        \includegraphics[width=\textwidth]{imagenes/pruebasconusuarios2.jpeg}
        \caption{Sesión de prueba 2.}
    \end{subfigure}
    \hfill
    \begin{subfigure}[b]{0.32\textwidth}
        \centering
        \includegraphics[width=\textwidth]{imagenes/pruebasconusuarios3.jpeg}
        \caption{Sesión de prueba 3.}
    \end{subfigure}
    \hfill
    \begin{subfigure}[b]{0.32\textwidth}
        \centering
        \includegraphics[width=\textwidth]{imagenes/pruebasconusuarios4.jpeg}
        \caption{Sesión de prueba 4.}
    \end{subfigure}
    \caption{Sesiones adicionales de pruebas de usabilidad realizadas con participantes en instalaciones universitarias.}
    \label{fig:anexo-pruebas-adicionales}
\end{figure}

\clearpage

\section{Anexo B: Cronograma del proyecto}

El siguiente cronograma muestra las fases de desarrollo del proyecto, desde la investigación inicial hasta la documentación final del trabajo de graduación.

\begin{figure}[htbp]
    \centering
    \includegraphics[width=\textwidth]{imagenes/cronograma.png}
    \caption{Cronograma de actividades del proyecto de desarrollo de la aplicación de realidad aumentada para Kaminaljuyú.}
    \label{fig:anexo-cronograma}
\end{figure}

\clearpage

\section{Anexo C: Repositorio del código fuente}

El código fuente completo de la aplicación de realidad aumentada para el Parque Arqueológico Kaminaljuyú se encuentra disponible públicamente como contribución a la comunidad de desarrollo de aplicaciones para patrimonio cultural. El repositorio incluye la implementación completa de todas las funcionalidades descritas en este documento, desde la integración con ARCore Geospatial API hasta la interfaz de usuario diseñada bajo principios de UX/UI.

El repositorio contiene:

\begin{itemize}
    \item Código fuente completo de la aplicación Android desarrollada en Kotlin
    \item Implementación de la integración con ARCore Geospatial API
    \item Componentes de interfaz de usuario y navegación
    \item Modelos 3D de las estructuras arqueológicas
    \item Recursos visuales y de contenido informativo
    \item Documentación técnica del proyecto
    \item Instrucciones detalladas de compilación y configuración
    \item Guías de instalación y despliegue
\end{itemize}

Este repositorio permite que otros desarrolladores, investigadores o instituciones culturales interesadas puedan estudiar la implementación, adaptarla a contextos similares o contribuir con mejoras al proyecto.

\subsection*{Acceso al repositorio}

El repositorio está alojado en la plataforma GitHub bajo licencia de código abierto:

\begin{center}
\texttt{https://github.com/mvrcentes/AR-Tour-Kaminaljuyu}
\end{center}

Se recomienda consultar el archivo README.md del repositorio para obtener información actualizada sobre requisitos del sistema, dependencias necesarias y procedimientos de instalación. El repositorio incluye también documentación sobre la arquitectura del código y comentarios explicativos que facilitan la comprensión de las decisiones técnicas implementadas.