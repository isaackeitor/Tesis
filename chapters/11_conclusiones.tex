% !TEX root = ../main.tex

\begin{itemize}
    \item Se logró diseñar e implementar exitosamente una aplicación de recorrido con realidad aumentada que integra una interfaz intuitiva y funcional. La tasa de completitud del 100\% en todas las tareas críticas evaluadas con 17 participantes confirma que la aplicación cumple con los criterios de usabilidad y accesibilidad establecidos. La implementación técnica mediante ARCore Geospatial API permitió crear una experiencia funcional que fomenta la interacción con el patrimonio arqueológico, logrando un Net Promoter Score de 47 en navegación entre secciones, lo que indica una aceptación positiva por parte de los usuarios.

    \item Se cumplió con el análisis de las características y necesidades de los visitantes del parque a través de investigación de usuarios, definición de personas y escenarios de uso. Este análisis permitió establecer requerimientos de diseño claros que guiaron el desarrollo de prototipos y la implementación final. Los resultados de las pruebas de usabilidad validaron que los perfiles identificados (visitantes con diversos niveles de familiaridad tecnológica) pudieron utilizar la aplicación exitosamente, confirmando la efectividad del análisis inicial en la definición de requerimientos.

    \item Se investigaron y aplicaron exitosamente principios de diseño centrado en el usuario y patrones de diseño específicos para aplicaciones móviles de realidad aumentada. La evolución del NPS desde 17 en la pantalla inicial hasta 47 en navegación evidencia la aplicación efectiva de estos principios. El uso de elementos visuales estándares (flechas direccionales 3D, gestos de manipulación convencionales) demuestra la implementación de mejores prácticas que facilitaron la curva de aprendizaje de los usuarios.

    \item Se lograron probar y validar iterativamente los prototipos y la implementación técnica mediante pruebas con usuarios reales. Las evaluaciones cuantitativas (System Usability Scale, Net Promoter Score, escalas Likert) y cualitativas proporcionaron datos concretos que permitieron identificar fortalezas y áreas de mejora. La tasa de completitud del 100\% valida la usabilidad funcional, mientras que los NPS progresivos demuestran la efectividad del proceso iterativo de refinamiento. Se identificaron áreas de mejora específicas que incluyen: la optimización de la pantalla principal para mejorar la orientación inicial (NPS de 17 indica espacio de mejora); la utilidad percibida del mapa en miniatura (64.7\%, por debajo del umbral esperado de 70\%); la claridad geométrica y proporciones de modelos 3D (35.3\% de evaluación positiva); y la incorporación de elementos multimedia adicionales (audio narrativo, videos interpretativos) que enriquezcan la experiencia educativa.

    \item Se desarrolló documentación completa del proceso que incluye recomendaciones técnicas, de diseño y de contenido que promueven la preservación y divulgación cultural de Kaminaljuyú. Las recomendaciones abarcan mejoras en modelado 3D, optimización de la interfaz inicial, y estrategias de implementación que pueden replicarse en otros sitios patrimoniales. Este trabajo establece un marco metodológico que combina necesidades tecnológicas con objetivos de preservación cultural, proporcionando un precedente para proyectos similares en contextos de patrimonio guatemalteco.
\end{itemize}
