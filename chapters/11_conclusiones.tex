% !TEX root = ../main.tex

Este trabajo de graduación abordó el diseño y desarrollo de una aplicación de realidad aumentada para el Parque Arqueológico Kaminaljuyú, aplicando metodologías de diseño centrado en el usuario y principios de UX/UI. Los resultados obtenidos permiten establecer las siguientes conclusiones:

La aplicación de metodologías de diseño centrado en el usuario demostró ser efectiva para crear una experiencia de realidad aumentada accesible y funcional. Las pruebas de usabilidad con 17 participantes revelaron una tasa de completitud del 100\% en todas las tareas críticas evaluadas, confirmando que el diseño logró hacer accesibles las funcionalidades esenciales para usuarios con diferentes niveles de familiaridad tecnológica. Este resultado valida que la inversión en investigación de usuarios, prototipado iterativo y evaluación continua produce aplicaciones que cumplen efectivamente con las necesidades de sus usuarios objetivo.

La evolución del Net Promoter Score desde 17 en la pantalla inicial hasta 47 en la navegación entre secciones evidencia una curva de aprendizaje positiva donde el diseño mejora conforme el usuario avanza en la interacción. Esta progresión demuestra que aunque la comunicación inicial presenta áreas de mejora, la estructura fundamental de navegación y las interacciones principales fueron diseñadas de manera coherente e intuitiva. La reducción de detractores de 3 a 1 y el aumento de promotores de 6 a 9 confirman que una vez superada la orientación inicial, los usuarios encuentran el sistema predecible y fácil de usar.

El sistema de navegación mediante flechas direccionales 3D implementado resultó efectivo para guiar a los usuarios en entornos amplios de realidad aumentada al aire libre. Las evaluaciones consistentemente positivas en tareas de navegación entre montículos validan este enfoque de diseño como solución práctica a los desafíos inherentes de orientación espacial en experiencias AR geoespaciales. Este hallazgo representa una contribución significativa al diseño de aplicaciones similares para patrimonio cultural, demostrando que elementos visuales simples pero consistentes pueden resolver problemas complejos de orientación sin sobrecargar la interfaz.

La interacción con modelos tridimensionales mediante gestos estándar de manipulación fue comprendida y ejecutada exitosamente por la mayoría de participantes, confirmando la importancia de adherirse a convenciones establecidas en lugar de implementar controles personalizados. Las evaluaciones positivas en tareas de exploración de modelos 3D demuestran que los usuarios pueden aprovechar su conocimiento previo de interacciones táctiles comunes, reduciendo la carga cognitiva asociada al aprendizaje de la aplicación.

Sin embargo, los resultados también identificaron aspectos específicos que requieren optimización. El mapa en miniatura, considerado útil solo por el 64.7\% de usuarios, y la claridad geométrica de modelos 3D, evaluada positivamente por apenas 35.3\% de participantes, representan oportunidades claras de mejora que podrían elevar significativamente la percepción general de calidad. Estos hallazgos subrayan que la usabilidad funcional, aunque necesaria, no es suficiente; la percepción de valor y calidad visual son igualmente importantes para una experiencia de usuario completa.

La integración exitosa de ARCore Geospatial API con un diseño de interfaz intuitivo demuestra la viabilidad técnica de crear aplicaciones AR patrimoniales sofisticadas pero accesibles. La abstracción de complejidades tecnológicas mediante elementos visuales directos permitió que usuarios sin experiencia previa en realidad aumentada pudieran completar todas las tareas evaluadas, validando el enfoque de priorizar simplicidad en la experiencia de usuario sobre exhibición de capacidades técnicas.

Este proyecto establece un precedente metodológico para el desarrollo de aplicaciones de realidad aumentada aplicadas al patrimonio cultural guatemalteco. La documentación del proceso completo, desde investigación de usuarios hasta validación mediante métricas cuantificables, proporciona un marco replicable que puede adaptarse a otros sitios arqueológicos o museos. Los datos empíricos obtenidos, incluyendo valores de NPS, evaluaciones Likert y tasas de completitud, constituyen referencias valiosas para proyectos futuros en contextos similares.

Finalmente, este trabajo confirma que la realidad aumentada representa una herramienta viable y efectiva para enriquecer la experiencia de visitantes en sitios patrimoniales. La tecnología, cuando se implementa con enfoque en necesidades humanas y se diseña siguiendo principios establecidos de usabilidad, puede hacer accesibles reconstrucciones arqueológicas y contextos históricos de manera que complementa y enriquece la experiencia física del sitio sin reemplazarla. El éxito medido en términos de completitud de tareas y evaluaciones positivas de facilidad de uso valida el potencial de aplicaciones AR como herramientas educativas para patrimonio cultural.
