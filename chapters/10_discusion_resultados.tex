% !TEX root = ../main.tex

\section{Interpretación de los hallazgos}

Los resultados de las pruebas de usabilidad realizadas con 17 participantes revelan aspectos importantes sobre la efectividad del diseño centrado en el usuario aplicado a la aplicación de realidad aumentada para el Parque Arqueológico Kaminaljuyú. La evaluación mediante Net Promoter Score y escalas Likert proporcionó datos cuantitativos que permiten comprender tanto las fortalezas como las áreas de oportunidad del diseño implementado.

El primer hallazgo significativo se relaciona con la orientación inicial de los usuarios en la aplicación. El NPS de 17 obtenido en la evaluación de expectativas desde la pantalla principal, con una distribución de 6 promotores, 8 pasivos y 3 detractores, indica que si bien la mayoría de usuarios pudo encontrar lo esperado, existe un margen considerable de mejora en la claridad de la navegación inicial. Este resultado sugiere que la pantalla principal, aunque funcional, no comunica de manera óptima las posibilidades y el flujo de interacción que ofrece la aplicación. La presencia de 8 usuarios pasivos implica que la experiencia inicial fue neutral o satisfactoria sin ser excepcional, lo cual representa una oportunidad para refinar los elementos visuales y textuales que guían al usuario en sus primeros momentos de interacción.

Contrasta notablemente con este primer resultado la evaluación de facilidad para navegar entre secciones, que alcanzó un NPS de 47 con 9 promotores, 7 pasivos y apenas 1 detractor. Esta mejora sustancial demuestra que una vez que los usuarios comprendieron la estructura de la aplicación y comenzaron a interactuar con ella, la navegación se volvió considerablemente más intuitiva. La reducción de detractores de 3 a 1 y el aumento de promotores de 6 a 9 evidencia una curva de aprendizaje positiva donde el diseño de las transiciones entre secciones resultó coherente y predecible. Este patrón sugiere que el problema principal no radica en la complejidad inherente de la interfaz, sino en la comunicación inicial de cómo utilizarla.

La evaluación de facilidad para completar tareas específicas del recorrido proporciona evidencia adicional sobre la efectividad del diseño implementado. Las seis tareas evaluadas muestran distribuciones consistentemente positivas en la escala de 1 a 5, con concentración mayoritaria en las categorías de facilidad y muy fácil. La tarea de iniciar el recorrido en el punto de partida fue percibida como clara e intuitiva por la mayoría de participantes, validando que el diseño del punto de inicio cumple su propósito de orientar al usuario sobre dónde comenzar la experiencia. Similarmente, la navegación al primer montículo usando las flechas direccionales 3D recibió evaluaciones predominantemente positivas, confirmando que este elemento de diseño funciona efectivamente como mecanismo de guía espacial.

La interacción con los modelos tridimensionales de los montículos representa uno de los aspectos técnicamente más complejos de la aplicación, involucrando gestos táctiles como pinch para zoom y swipe para rotación. Las evaluaciones mayormente positivas en esta tarea demuestran que los usuarios pudieron comprender y ejecutar estos gestos sin dificultad significativa, lo cual valida la decisión de utilizar gestos estándar de manipulación en lugar de controles personalizados. Esta comprensibilidad intuitiva de la interacción con modelos 3D es particularmente relevante considerando que no todos los usuarios tienen experiencia previa con aplicaciones de realidad aumentada.

El acceso a información detallada de cada montículo fue evaluado consistentemente como fácil y muy fácil, confirmando que el botón de información y la estructura de los diálogos informativos son suficientemente evidentes y accesibles. Este resultado es crucial desde la perspectiva del objetivo educativo de la aplicación, ya que garantiza que los usuarios pueden acceder al contenido arqueológico e histórico sin barreras de usabilidad. La facilidad percibida para avanzar al siguiente montículo y para finalizar el recorrido retornando al punto de inicio completa un panorama positivo sobre la fluidez del recorrido completo.

Un aspecto particularmente notable de los resultados es la tasa de completitud del 100\% en todas las tareas críticas evaluadas. Este indicador demuestra que, independientemente de las variaciones en la facilidad percibida, todos los participantes lograron exitosamente ejecutar cada una de las funciones esenciales de la aplicación. Esta completitud universal contrasta favorablemente con muchas aplicaciones móviles complejas donde es común que algunos usuarios abandonen tareas o requieran asistencia externa. El hecho de que ningún participante fallara en completar alguna tarea sugiere que el diseño alcanzó un nivel funcional de usabilidad donde las características esenciales son accesibles para usuarios con diferentes niveles de familiaridad tecnológica.

Sin embargo, los resultados también revelaron áreas específicas que requieren atención en futuras iteraciones del diseño. El mapa en miniatura, a pesar de estar implementado como herramienta de orientación, fue considerado útil solo por el 64.7\% de los usuarios, quedando por debajo del umbral objetivo del 70\%. Esta discrepancia entre la intención de diseño y la percepción de utilidad sugiere que aunque el elemento existe y es funcionalmente correcto, su presentación visual, tamaño, ubicación en la interfaz o nivel de detalle no están comunicando efectivamente su propósito al usuario. La miniaturización excesiva puede dificultar la interpretación de la información espacial, o bien la falta de elementos contextuales claros puede hacer que los usuarios no reconozcan inmediatamente el valor de esta funcionalidad para su orientación.

La claridad geométrica y las proporciones de los modelos 3D constituyen otra área crítica identificada, con solo 35.3\% de evaluaciones positivas. Este resultado contrasta con la evaluación favorable del valor educativo general del contenido y la coherencia visual entre modelos, sugiriendo que el problema no es conceptual sino de ejecución técnica. Los usuarios reconocen que los modelos cumplen un propósito educativo y que existe consistencia en el estilo visual, pero perciben que la representación geométrica individual podría ser más clara y detallada. Esta observación indica que el refinamiento de la geometría de los modelos, la mejora de texturas y la incorporación de elementos de escala visual podrían elevar significativamente la percepción de calidad sin necesidad de modificar el enfoque general de diseño.

\section{Implicaciones de los resultados}

Los hallazgos de este trabajo tienen implicaciones importantes tanto para el desarrollo futuro de esta aplicación específica como para proyectos similares de realidad aumentada aplicada al patrimonio cultural. La progresión observada en los valores de NPS desde 17 en la pantalla inicial hasta 47 en la navegación entre secciones demuestra que el diseño de la experiencia de usuario mejora conforme el usuario avanza en la interacción. Esta progresión sugiere que invertir esfuerzo adicional en mejorar la comunicación y orientación en la pantalla principal podría elevar significativamente la percepción general de la aplicación desde el primer contacto.

La efectividad demostrada del sistema de flechas direccionales 3D como mecanismo de navegación espacial en entornos de realidad aumentada al aire libre representa un aporte significativo al diseño de este tipo de aplicaciones. La navegación en exteriores amplios mediante realidad aumentada presenta desafíos particulares relacionados con la precisión del GPS, la orientación del usuario en el espacio físico y la comprensión de la dirección hacia objetivos no inmediatamente visibles. Los resultados positivos obtenidos validan que un sistema visual simple pero consistente puede resolver estos desafíos de manera efectiva sin requerir elementos de interfaz complejos o sobrecargados.

La tasa de completitud del 100\% en todas las tareas críticas sugiere que el enfoque metodológico de diseño centrado en el usuario, que incluyó investigación de usuarios, prototipado iterativo y pruebas de usabilidad, fue efectivo para identificar y resolver problemas potenciales antes del lanzamiento. Esta efectividad metodológica tiene implicaciones prácticas importantes para instituciones culturales o equipos de desarrollo con recursos limitados, demostrando que es posible alcanzar altos niveles de usabilidad sin presupuestos extensivos si se aplican rigurosamente principios establecidos de diseño UX/UI.

La identificación de áreas problemáticas específicas como el mapa en miniatura y la claridad geométrica de modelos 3D tiene valor tanto para la mejora de esta aplicación como para el aprendizaje sobre diseño de AR patrimonial en general. El caso del mapa en miniatura ilustra que la simple inclusión de una funcionalidad potencialmente útil no garantiza que los usuarios la perciban como valiosa si su diseño visual o integración en la interfaz no comunica claramente su propósito. Esta lección refuerza la importancia de no asumir que elementos que son estándar en aplicaciones de navegación tradicionales serán automáticamente comprendidos en contextos de realidad aumentada, donde la atención del usuario está distribuida entre el mundo físico y la capa digital.

El desafío identificado con la claridad geométrica de los modelos 3D plantea una tensión inherente en proyectos de reconstrucción arqueológica mediante tecnologías digitales. Por un lado, existe la necesidad de representaciones precisas y detalladas que comuniquen efectivamente la escala, proporciones y características arquitectónicas de las estructuras originales. Por otro lado, limitaciones técnicas relacionadas con el rendimiento de dispositivos móviles, tamaños de archivos y capacidades de renderizado en tiempo real imponen restricciones sobre el nivel de detalle alcanzable. Los resultados sugieren que el balance actual entre detalle y rendimiento podría optimizarse, posiblemente mediante técnicas de simplificación geométrica más sofisticadas que preserven la claridad visual mientras mantienen eficiencia computacional.

\section{Contribuciones del trabajo}

Este proyecto aporta conocimiento tanto metodológico como práctico al campo emergente de aplicaciones de realidad aumentada para patrimonio cultural. Desde el punto de vista metodológico, la documentación del proceso completo de diseño centrado en el usuario aplicado específicamente a AR patrimonial proporciona un marco de referencia para proyectos similares. La secuencia de investigación de usuarios mediante observaciones de campo y entrevistas, definición de requerimientos basados en necesidades reales, diseño iterativo de prototipos, implementación técnica con enfoque en UX/UI, y validación mediante pruebas de usabilidad con métricas cuantificables, constituye un protocolo replicable que puede adaptarse a otros contextos de patrimonio cultural.

Los datos cuantitativos obtenidos establecen valores de referencia para la evaluación de aplicaciones AR patrimoniales en contextos similares. El NPS de 47 para navegación, las evaluaciones de facilidad de tareas en escala Likert, y particularmente la tasa de completitud del 100\% en tareas críticas, proporcionan puntos de comparación para futuros desarrollos en Guatemala y la región mesoamericana. Esta contribución empírica es valiosa considerando la escasez relativa de estudios cuantitativos rigurosos sobre usabilidad de AR patrimonial en contextos latinoamericanos.

Desde una perspectiva técnica, la integración exitosa de ARCore Geospatial API con un diseño de interfaz intuitivo demuestra la viabilidad de implementar tecnologías avanzadas de posicionamiento geoespacial manteniendo simplicidad en la experiencia de usuario. La abstracción de complejidades técnicas mediante elementos visuales directos como flechas direccionales, indicadores de proximidad y botones claramente identificados representa un modelo de diseño que equilibra sofisticación tecnológica con accesibilidad de uso.

El desarrollo de contenido arqueológico estructurado para 11 puntos de interés, incluyendo modelos 3D de estructuras y fichas informativas detalladas, constituye un activo digital reutilizable que puede servir como base para futuras ampliaciones o adaptaciones de la aplicación. La estructura modular de este contenido facilita su actualización o enriquecimiento sin requerir modificaciones fundamentales de la arquitectura de la aplicación.

\section{Limitaciones identificadas}

Es importante reconocer las limitaciones de este trabajo para contextualizar adecuadamente sus hallazgos y conclusiones. Las pruebas de usabilidad se realizaron con 17 participantes en un entorno universitario controlado, no en el parque arqueológico real. Este contexto de prueba, aunque útil para evaluar la usabilidad fundamental de la interfaz, no captura completamente las condiciones reales de uso como iluminación solar variable, condiciones climáticas cambiantes, ruido ambiental urbano y fatiga física de caminatas prolongadas. La usabilidad observada en este contexto controlado podría no replicarse completamente cuando los usuarios enfrentan las complejidades adicionales del entorno real del parque.

La duración de las sesiones de prueba, aproximadamente 15 a 20 minutos, es significativamente menor que la duración estimada de un recorrido completo real del parque que podría tomar entre 1 y 2 horas. Esta diferencia temporal implica que efectos potenciales como fatiga visual por uso prolongado de realidad aumentada, consumo de batería del dispositivo, o cansancio físico acumulado no fueron adecuadamente evaluados. La experiencia de usuario podría degradarse con el tiempo de uso, aspecto que no se refleja en las métricas obtenidas en sesiones breves.

La muestra de participantes, aunque adecuada para identificar problemas mayores de usabilidad, presenta un sesgo hacia población universitaria joven con probable alta familiaridad tecnológica. Los visitantes reales del Parque Arqueológico Kaminaljuyú incluyen familias con niños, adultos mayores, turistas internacionales y personas con diferentes niveles de experiencia tecnológica. Los resultados de usabilidad podrían variar con poblaciones demográficamente más diversas, particularmente con usuarios de mayor edad o con menor exposición a tecnologías móviles.

La precisión histórica y arqueológica de los modelos 3D y del contenido textual, aunque basada en fuentes bibliográficas consultadas, no fue sometida a validación formal por arqueólogos especializados en Kaminaljuyú. Esta limitación implica que mientras la usabilidad y efectividad de la interfaz fueron rigurosamente evaluadas, la credibilidad científica del contenido arqueológico representado depende de la calidad de las fuentes secundarias utilizadas más que de validación experta directa.

El proyecto se limitó a la plataforma Android, excluyendo aproximadamente el 30\% del mercado guatemalteco que utiliza dispositivos iOS. Esta restricción de plataforma, aunque justificada por las capacidades específicas de ARCore Geospatial API y las limitaciones de recursos de desarrollo, reduce significativamente el alcance potencial de la aplicación y limita la diversidad de usuarios que pueden acceder a ella.
