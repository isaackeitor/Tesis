% !TEX root = z-main.tex

Esta investigación siguió una metodología cualitativa basada en el enfoque de diseño centrado en el usuario. Este enfoque buscó que algunas personas que usarían la aplicación participaran en las distintas etapas del proyecto, para asegurar que el resultado final se adaptara bien a sus necesidades, expectativas y capacidades. El objetivo fue crear una aplicación de recorrido con Realidad Aumentada (RA) para el Parque Arqueológico Kaminaljuyú que fuera fácil de usar, llamativa y útil. A continuación, se describen las etapas del proceso:

\section{Investigación de usuarios}

La primera etapa consistió en conocer mejor a los visitantes del parque y entender cómo era actualmente su experiencia. Para ello se utilizaron dos métodos principales:

\begin{itemize}
    \item \textbf{Observaciones en campo}, donde se simuló de forma personal la experiencia de visitante del parque para validar rutas potenciales de recorrido, identificar elementos del entorno que atraen la atención y comprender cómo los visitantes podrían interactuar con el espacio arqueológico. Esto se realizó mediante visitas al parque en las que se adoptó el comportamiento de un visitante común para determinar qué aspectos del sitio resultaban más interesantes y cómo se desarrollaría naturalmente la interacción con el entorno.
    \item \textbf{Entrevistas semiestructuradas}, que se realizaron a distintos estudiantes durante las sesiones de pruebas de usabilidad en la universidad para conocer qué esperaban de la aplicación, qué les interesaba, qué dificultades encontraban y qué opinaban de la experiencia propuesta.
\end{itemize}

Luego, toda esta información se organizó y se buscaron patrones o ideas comunes que ayudaron a definir mejor el diseño de la aplicación.

\section{Definición de requerimientos}

Con base en los hallazgos anteriores, se establecieron los requerimientos que debía cumplir la aplicación. Estos se dividieron en:

\begin{itemize}
    \item \textbf{Requerimientos funcionales}, es decir, lo que la aplicación debía hacer.
    \item \textbf{Requerimientos de diseño}, que tenían que ver con la experiencia de uso y la apariencia.
\end{itemize}

\section{Diseño de prototipos}

En esta fase se realizaron versiones iniciales del diseño de la aplicación, primero más simples (baja fidelidad) y luego más completas (alta fidelidad), usando herramientas de diseño digital. Esto incluyó:

\begin{itemize}
    \item La organización del contenido y navegación.
    \item El diseño visual, eligiendo colores, letras y botones que transmitieran sensaciones como calma, interés o claridad.
    \item La simulación de recorridos, para imaginar cómo usaría una persona la app durante su visita y comprobar que todo estuviera bien ordenado.
\end{itemize}

\section{Implementación técnica}

Esta etapa abarcó tanto la programación de funcionalidades de realidad aumentada como el diseño e implementación de la interfaz de usuario (UI) y la experiencia de usuario (UX). Se desarrolló la aplicación utilizando la API Geoespacial de ARCore para posicionar modelos 3D en ubicaciones reales del parque. Paralelamente, se trabajó en el diseño de componentes visuales intuitivos, sistemas de navegación claros, flujos de interacción coherentes y retroalimentación visual consistente que facilitaran la comprensión y el uso efectivo de la tecnología AR por parte de usuarios diversos. Se prestó especial atención a la jerarquía visual de elementos en pantalla, la legibilidad de textos en condiciones de iluminación variable, la accesibilidad de controles táctiles, la consistencia de patrones de diseño a lo largo de la aplicación y la optimización del rendimiento para asegurar una experiencia fluida y sin interrupciones.

\section{Pruebas de usabilidad}

Durante y después del desarrollo se realizaron pruebas con usuarios reales para comprobar si la aplicación era clara, útil y agradable. Se llevaron a cabo:

\begin{itemize}
    \item \textbf{Pruebas en la universidad}, con estudiantes que usaron la aplicación en sesiones controladas. La realización de pruebas en instalaciones universitarias respondió a razones de practicidad logística, seguridad y control de variables del entorno de evaluación.
    \item \textbf{Observaciones y entrevistas} después del uso para saber qué funcionó y qué se podía mejorar.
    \item \textbf{Revisión de métricas de usabilidad}, como cuánto tiempo usaban la app, si lograban completar las tareas asignadas, qué dificultades encontraban y qué elementos de la interfaz resultaban confusos o efectivos.
\end{itemize}

Los resultados permitieron hacer cambios y mejorar el diseño antes de finalizar el proyecto.

\section{Elaboración del trabajo de graduación}

Como cierre del proyecto, se elaboró el presente trabajo escrito de graduación que documenta y analiza:

\begin{itemize}
    \item Todo el proceso de diseño y desarrollo de la aplicación, desde la investigación inicial hasta la implementación final.
    \item Las prácticas de UX/UI que demostraron ser efectivas en el contexto de realidad aumentada para patrimonio cultural.
    \item Recomendaciones para futuras versiones de la aplicación y guías para aplicar soluciones similares en otros sitios arqueológicos o culturales.
\end{itemize}