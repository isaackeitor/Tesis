% !TEX root = z-main.tex

La preservación y divulgación del patrimonio arqueológico requerían de enfoques innovadores que lograran conectar con las audiencias contemporáneas. En el caso del Parque Arqueológico Kaminaljuyú, uno de los sitios más relevantes de la civilización maya preclásica, persistía una brecha entre el valor histórico del lugar y la manera en que dicho valor era percibido y comprendido por el visitante promedio. Esta desconexión se debía, en gran parte, a la falta de herramientas interpretativas accesibles, visuales y dinámicas que facilitaran la apropiación del conocimiento cultural. Como señalaron \cite{gutierrez2011visor}, la ausencia de recursos virtuales integrados con los elementos reales generaba una significativa brecha interpretativa.

La Realidad Aumentada (RA) había emergido como una tecnología capaz de transformar la interacción entre los usuarios y su entorno físico, integrando contenidos virtuales que enriquecían la experiencia educativa, cultural y turística. Su aplicación en espacios patrimoniales no solo permitía visualizar estructuras antiguas o reconstrucciones históricas, sino que también favorecía la preservación digital del legado cultural y su difusión entre públicos más jóvenes y tecnológicamente familiarizados \cite{komianos2024introducing}.

En este contexto, el presente proyecto propuso el diseño e implementación de una aplicación de recorrido con RA que respondiera a principios de diseño centrado en el usuario. La iniciativa no se limitó a incorporar tecnología de forma superficial, sino que priorizó la creación de una experiencia coherente y funcional a través de una interfaz de usuario (UI) intuitiva y estética, pensada para facilitar la navegación, el acceso al contenido y la conexión emocional con el sitio arqueológico. Al mismo tiempo, el sistema debía proporcionar a los visitantes del museo o espacio patrimonial una interfaz intuitiva de manera que los usuarios pudieran interactuar con los contenidos digitales de manera fácil y natural, así como lo harían con objetos en el mundo real \cite{wojciechowski2004building}.

Asimismo, el enfoque metodológico del proyecto contempló la iteración constante mediante pruebas de usabilidad con usuarios reales, lo cual permitió validar y mejorar las decisiones de diseño basadas en datos empíricos. Como recomendaron \cite{quinones2018methodology}, complementar las evaluaciones heurísticas con tests de usuario y refinar el sistema según la retroalimentación obtenida aseguró que la solución final no solo fuera técnicamente viable, sino también relevante, inclusiva y culturalmente pertinente. De esta manera, se contribuyó tanto al fortalecimiento de la experiencia museográfica del parque como al desarrollo de nuevas formas de acceso al patrimonio mediante tecnología accesible, pedagógica y atractiva.