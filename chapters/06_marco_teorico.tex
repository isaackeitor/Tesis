% !TEX root = z-main.tex

\section{Realidad Aumentada como Herramienta para la Mediación del Patrimonio Cultural}

La Realidad Aumentada (RA) se había consolidado como una herramienta poderosa para la interpretación y mediación del patrimonio cultural. Su capacidad para superponer elementos digitales en tiempo real sobre el entorno físico permitía ofrecer nuevas formas de explorar, aprender y conectar con sitios históricos. En particular, permitía crear experiencias interactivas que iban más allá de la simple observación, fomentando una participación activa del visitante en la reconstrucción y comprensión del pasado. Estas experiencias, al combinar información textual, visual y auditiva, promovían un aprendizaje multisensorial que respondía a las distintas formas de percepción y comprensión del usuario.

En sitios como Kaminaljuyú, donde gran parte de las estructuras originales habían sido erosionadas por el tiempo o estaban fragmentadas, la RA cumplía una función de "restauración virtual", al posibilitar la visualización de contextos arqueológicos en su estado original o hipotético. Esto no solo enriquecía la visita, sino que también democratizaba el acceso al conocimiento al eliminar barreras interpretativas que solían estar presentes en espacios museográficos tradicionales. Sin embargo, la eficacia de la RA como mediadora cultural no se encontraba únicamente en su capacidad tecnológica, sino en el modo en que esta tecnología se diseñaba e implementaba para el usuario final \cite{boboc2022augmented}. La mediación efectiva requería una narrativa cohesionada, una integración visual armónica con el entorno y una accesibilidad funcional que contemplara la diversidad de perfiles de los visitantes.

\section{Principios de Experiencia de Usuario (UX) y Diseño de Interfaz de Usuario (UI)}

La integración exitosa de tecnología RA en entornos patrimoniales no podía desvincularse del diseño centrado en el usuario (DCU), ya que el objetivo principal era facilitar la comprensión del contenido y generar una experiencia significativa. Este enfoque reconocía que las personas interactuaban con los sistemas tecnológicos desde diversos niveles de conocimiento, expectativas y habilidades. Por ello, diseñar una experiencia de usuario eficaz implicaba considerar desde el inicio aspectos como la accesibilidad, la claridad del flujo de navegación, el tiempo de respuesta del sistema y la estética de la interfaz.

En este contexto, el diseño de UX se enfocaba en garantizar que la interacción fuera intuitiva, agradable y que mantuviera la atención del visitante, mientras que el diseño de UI se encargaba de traducir esa experiencia en elementos visuales consistentes y funcionales. El equilibrio entre estos dos componentes era crucial, especialmente en contextos donde el usuario se desplazaba físicamente por el espacio, como ocurría en parques arqueológicos \cite{poux2020initial}.

Para el caso de Kaminaljuyú, esto implicaba una interfaz que no solo fuera informativa y visualmente armónica, sino que también se adaptara al entorno físico real. Esto se logró mediante el uso de íconos reconocibles, navegación guiada basada en la geolocalización del usuario, alertas contextuales sobre puntos de interés y la personalización del contenido según el perfil del visitante (por ejemplo, nivel educativo o idioma). Según \cite{garrett2011elements}, una arquitectura de la información bien estructurada era esencial para evitar la sobrecarga cognitiva y facilitar una experiencia fluida. Asimismo, \cite{ramtohul2023user} destacaron la importancia de integrar recorridos optimizados que respondieran a distintos intereses, como el histórico, educativo o turístico.

\section{Contexto Espacial y Diseño Interactivo}

En aplicaciones de RA, el diseño no se limitaba a lo visual: se trataba también de definir cómo y cuándo ocurría la interacción con el contenido digital. La colocación espacial de objetos aumentados, el momento exacto de aparición y la forma en que se invitaba al usuario a interactuar eran aspectos determinantes para una experiencia exitosa. Aquí entraba en juego el diseño contextual, que permitía adaptar la experiencia a las condiciones específicas del entorno —iluminación, ruido, obstáculos físicos— y del usuario —como edad, movilidad, conocimiento previo.

Tecnologías como ARCore Geospatial API permitían anclar contenido digital en ubicaciones geográficas precisas, pero su verdadero valor se manifestaba cuando esa precisión se traducía en una experiencia narrativa coherente. Por ejemplo, al llegar a una estructura parcialmente conservada, el sistema podía superponer una reconstrucción digital solo si detectaba que el usuario estaba en el ángulo adecuado y que había completado cierta secuencia lógica del recorrido. Este tipo de microinteracciones no solo enriquecían la experiencia, sino que también fomentaban una exploración activa y no lineal del sitio \cite{komianos2024introducing}.

Un diseño bien logrado lograba que el contenido digital pareciera una extensión natural del entorno, sin distracciones ni rupturas en la experiencia. Para ello, era fundamental que el sistema respondiera con fluidez, minimizara los errores de anclaje y se mantuviera funcional incluso con limitaciones de conectividad o condiciones ambientales adversas. El desafío no era únicamente tecnológico, sino también narrativo y humano: transformar datos arqueológicos en historias vivas, accesibles y memorables.