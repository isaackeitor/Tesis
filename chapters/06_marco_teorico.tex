% !TEX root = z-main.tex

\section{Conceptos Fundamentales}

Para comprender el alcance y las implicaciones de este trabajo, es necesario establecer con claridad los conceptos fundamentales que sustentan el diseño y desarrollo de aplicaciones de realidad aumentada orientadas al patrimonio cultural. Estos conceptos constituyen la base teórica sobre la cual se construye la propuesta y guían las decisiones de diseño e implementación.

\subsection{Realidad Aumentada}

La Realidad Aumentada (RA) se define como una tecnología que permite superponer información digital —ya sea visual o textual— sobre el entorno físico en tiempo real, creando una experiencia mixta entre lo real y lo virtual \cite{azuma1997survey}. A diferencia de la realidad virtual, que reemplaza completamente el entorno del usuario, la RA lo complementa al añadir capas de información contextual que enriquecen la percepción del mundo físico sin aislarlo de este.

Según \cite{azuma1997survey}, un sistema de RA debe cumplir con tres características esenciales: combinar elementos reales y virtuales, operar en tiempo real, y registrar correctamente los objetos virtuales en el espacio tridimensional del entorno físico. Estas capacidades permiten aplicaciones diversas que van desde el entretenimiento hasta la educación y el turismo cultural. En el contexto patrimonial, la RA facilita la reconstrucción virtual de sitios históricos, la narración contextualizada de eventos pasados y la visualización de información arqueológica que de otro modo sería inaccesible para el visitante promedio.

Tecnologías como ARCore de Google y ARKit de Apple han democratizado el acceso a experiencias de realidad aumentada en dispositivos móviles, permitiendo el desarrollo de aplicaciones sin necesidad de hardware especializado \cite{arcore2024documentation}. Esto ha resultado en una proliferación de soluciones de RA aplicadas a museos, sitios arqueológicos y espacios patrimoniales, donde se busca mejorar la experiencia del visitante y fomentar la educación a través de medios interactivos e inmersivos \cite{bekele2018survey}.

\subsection{Interfaz de Usuario (UI)}

La Interfaz de Usuario (UI, por sus siglas en inglés) se refiere al conjunto de elementos visuales, textuales y gráficos a través de los cuales un usuario interactúa con un sistema digital. En términos funcionales, la UI es el punto de contacto entre el ser humano y la tecnología, y su diseño determina en gran medida la facilidad con la que un usuario puede lograr sus objetivos al utilizar una aplicación \cite{shneiderman2016designing}.

El diseño de interfaces de usuario implica decisiones sobre la disposición visual de los elementos (layout), la jerarquía de la información, la tipografía, el uso del color, los íconos, los botones y otros componentes interactivos. Según \cite{shneiderman2016designing}, los principios fundamentales del diseño de UI incluyen la consistencia en el uso de elementos, la retroalimentación clara y oportuna al usuario, la prevención de errores y la simplicidad que permite reducir la carga cognitiva.

En el contexto de aplicaciones móviles de realidad aumentada, el diseño de UI enfrenta retos adicionales debido a la naturaleza híbrida del entorno: el usuario debe interactuar simultáneamente con elementos digitales y físicos, lo que exige interfaces minimalistas, no invasivas y contextualmente relevantes. Esto implica un balance cuidadoso entre mostrar información útil y no obstruir la vista del entorno real, un desafío particularmente importante en aplicaciones patrimoniales donde la experiencia del sitio físico debe preservarse \cite{mobileARpatterns2024}.

\subsection{Experiencia de Usuario (UX)}

La Experiencia de Usuario (UX, por sus siglas en inglés) es un concepto más amplio que la interfaz de usuario, ya que abarca todos los aspectos de la interacción de una persona con un producto, servicio o sistema \cite{garrett2011elements}. Don Norman, quien acuñó el término en los años noventa, la define como la totalidad de las percepciones, emociones y respuestas que resultan del uso o la anticipación del uso de un producto \cite{norman2013design}.

A diferencia de la UI, que se centra en los elementos visuales y funcionales de la interfaz, la UX considera dimensiones más profundas de la interacción: la utilidad del sistema (si resuelve una necesidad real), la usabilidad (cuán fácil es de usar), la accesibilidad (si puede ser utilizada por personas con diversas capacidades), el deseo (si genera una respuesta emocional positiva) y el valor percibido \cite{garrett2011elements}. El diseño de UX, por tanto, es un proceso interdisciplinario que integra investigación de usuarios, arquitectura de la información, diseño de interacción y evaluación continua.

En aplicaciones de realidad aumentada para patrimonio cultural, la experiencia de usuario no solo debe considerar la eficiencia de la navegación o la claridad de la información, sino también aspectos como el sentido de presencia, la inmersión narrativa, el aprendizaje significativo y la conexión emocional con el sitio histórico \cite{ramtohul2023user}. Una experiencia de usuario exitosa en este contexto no solo informa, sino que transforma la relación del visitante con el espacio cultural, generando memorias duraderas y fomentando el interés por la conservación del patrimonio.

\subsection{Diseño Centrado en el Usuario (DCU)}

El Diseño Centrado en el Usuario (DCU) es una filosofía y un conjunto de metodologías que sitúan al usuario final en el centro del proceso de diseño y desarrollo. Según la norma ISO 9241-210:2019 \cite{iso2019ergonomics}, el DCU se caracteriza por la participación activa de los usuarios a lo largo de todo el ciclo de desarrollo, la consideración explícita de sus necesidades y contextos de uso, y la realización de evaluaciones iterativas que permiten refinar el diseño en función del feedback recibido.

El DCU se basa en principios fundamentales que incluyen: entender profundamente a los usuarios y sus contextos, involucrarlos en el proceso de diseño, diseñar soluciones basadas en evidencia empírica, iterar constantemente a partir de evaluaciones de usabilidad, y adoptar un enfoque multidisciplinario que integre perspectivas de diseño, ingeniería, psicología y ciencias sociales \cite{norman2013design}.

En el contexto de aplicaciones de RA para patrimonio cultural, el DCU implica comprender no solo las habilidades tecnológicas de los visitantes, sino también sus motivaciones para visitar el sitio, sus conocimientos previos sobre el tema, sus expectativas de la experiencia, y las barreras físicas, cognitivas o culturales que podrían enfrentar \cite{poux2020initial}. Este enfoque permite diseñar experiencias que sean verdaderamente inclusivas y significativas, en lugar de imponer soluciones tecnológicas que no respondan a necesidades reales.

\section{Realidad Aumentada como Herramienta para la Mediación del Patrimonio Cultural}

La Realidad Aumentada (RA) se había consolidado como una herramienta poderosa para la interpretación y mediación del patrimonio cultural. Su capacidad para superponer elementos digitales en tiempo real sobre el entorno físico permitía ofrecer nuevas formas de explorar, aprender y conectar con sitios históricos. En particular, permitía crear experiencias interactivas que iban más allá de la simple observación, fomentando una participación activa del visitante en la reconstrucción y comprensión del pasado. Estas experiencias, al combinar información textual y visual, promovían un aprendizaje que respondía a las distintas formas de percepción y comprensión del usuario.

En sitios como Kaminaljuyú, donde gran parte de las estructuras originales habían sido erosionadas por el tiempo o estaban fragmentadas, la RA cumplía una función de "restauración virtual", al posibilitar la visualización de contextos arqueológicos en su estado original o hipotético. Esto no solo enriquecía la visita, sino que también democratizaba el acceso al conocimiento al eliminar barreras interpretativas que solían estar presentes en espacios museográficos tradicionales. Sin embargo, la eficacia de la RA como mediadora cultural no se encontraba únicamente en su capacidad tecnológica, sino en el modo en que esta tecnología se diseñaba e implementaba para el usuario final \cite{boboc2022augmented}. La mediación efectiva requería una narrativa cohesionada, una integración visual armónica con el entorno y una accesibilidad funcional que contemplara la diversidad de perfiles de los visitantes.

\section{Principios de Experiencia de Usuario (UX) y Diseño de Interfaz de Usuario (UI)}

La integración exitosa de tecnología RA en entornos patrimoniales no podía desvincularse del diseño centrado en el usuario (DCU), ya que el objetivo principal era facilitar la comprensión del contenido y generar una experiencia significativa. Este enfoque reconocía que las personas interactuaban con los sistemas tecnológicos desde diversos niveles de conocimiento, expectativas y habilidades. Por ello, diseñar una experiencia de usuario eficaz implicaba considerar desde el inicio aspectos como la accesibilidad, la claridad del flujo de navegación, el tiempo de respuesta del sistema y la estética de la interfaz.

En este contexto, el diseño de UX se enfocaba en garantizar que la interacción fuera intuitiva, agradable y que mantuviera la atención del visitante, mientras que el diseño de UI se encargaba de traducir esa experiencia en elementos visuales consistentes y funcionales. El equilibrio entre estos dos componentes era crucial, especialmente en contextos donde el usuario se desplazaba físicamente por el espacio, como ocurría en parques arqueológicos \cite{poux2020initial}.

Para el caso de Kaminaljuyú, esto implicaba una interfaz que no solo fuera informativa y visualmente armónica, sino que también se adaptara al entorno físico real. Esto se logró mediante el uso de íconos reconocibles, navegación guiada basada en la geolocalización del usuario, alertas contextuales sobre puntos de interés y la personalización del contenido según el perfil del visitante (por ejemplo, nivel educativo o idioma). Según \cite{garrett2011elements}, una arquitectura de la información bien estructurada era esencial para evitar la sobrecarga cognitiva y facilitar una experiencia fluida. Asimismo, \cite{ramtohul2023user} destacaron la importancia de integrar recorridos optimizados que respondieran a distintos intereses, como el histórico, educativo o turístico.

\section{Contexto Espacial y Diseño Interactivo}

En aplicaciones de RA, el diseño no se limitaba a lo visual: se trataba también de definir cómo y cuándo ocurría la interacción con el contenido digital. La colocación espacial de objetos aumentados, el momento exacto de aparición y la forma en que se invitaba al usuario a interactuar eran aspectos determinantes para una experiencia exitosa. Aquí entraba en juego el diseño contextual, que permitía adaptar la experiencia a las condiciones específicas del entorno —iluminación, ruido, obstáculos físicos— y del usuario —como edad, movilidad, conocimiento previo.

Tecnologías como ARCore Geospatial API permitían anclar contenido digital en ubicaciones geográficas precisas, pero su verdadero valor se manifestaba cuando esa precisión se traducía en una experiencia narrativa coherente. Por ejemplo, al llegar a una estructura parcialmente conservada, el sistema podía superponer una reconstrucción digital solo si detectaba que el usuario estaba en el ángulo adecuado y que había completado cierta secuencia lógica del recorrido. Este tipo de microinteracciones no solo enriquecían la experiencia, sino que también fomentaban una exploración activa y no lineal del sitio \cite{komianos2024introducing}.

Un diseño bien logrado lograba que el contenido digital pareciera una extensión natural del entorno, sin distracciones ni rupturas en la experiencia. Para ello, era fundamental que el sistema respondiera con fluidez, minimizara los errores de anclaje y se mantuviera funcional incluso con limitaciones de conectividad o condiciones ambientales adversas. El desafío no era únicamente tecnológico, sino también narrativo y humano: transformar datos arqueológicos en historias vivas, accesibles y memorables.