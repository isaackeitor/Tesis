% !TEX root = ../main.tex

Los hallazgos y limitaciones identificados sugieren direcciones claras para trabajo futuro que podría ampliar y mejorar este proyecto:

\begin{itemize}
    \item La optimización de la pantalla principal emerge como una prioridad inmediata dado el NPS relativamente bajo de 17. Esta mejora podría abordarse mediante técnicas como tutoriales interactivos breves al primer uso, elementos visuales más prominentes que guíen la atención hacia las funciones principales, o una reorganización de la jerarquía visual para hacer más evidentes las opciones de navegación disponibles.

    \item El rediseño del mapa en miniatura requiere investigación específica sobre cómo los usuarios interpretan representaciones espaciales en contextos de realidad aumentada. Posibles aproximaciones incluyen aumentar el tamaño del mapa, incorporar animaciones o resaltados que indiquen la posición actual del usuario y su relación con los objetivos, o integrar el mapa de manera más orgánica con los elementos de realidad aumentada en lugar de presentarlo como un componente separado de la interfaz.

    \item La mejora de la claridad geométrica de los modelos 3D podría abordarse mediante técnicas de optimización que mantengan detalles arquitectónicos clave mientras reducen polígonos en áreas menos visualmente importantes. La incorporación de texturas de mayor resolución en superficies prominentes, elementos de escala visual como figuras humanas de referencia, o indicadores dimensionales podría mejorar significativamente la percepción de claridad sin impactar negativamente el rendimiento.

    \item La realización de pruebas de usabilidad in situ en el parque arqueológico real constituye un siguiente paso metodológico importante. Estas pruebas permitirían validar que la usabilidad observada en condiciones controladas se mantiene en el entorno real de uso, identificar problemas específicos relacionados con iluminación solar, precisión GPS en diferentes áreas del parque, o desafíos de orientación en el espacio físico que no son evidentes en pruebas de laboratorio.

    \item La ampliación de la muestra de participantes para incluir mayor diversidad demográfica proporcionaría información valiosa sobre cómo diferentes perfiles de usuarios experimentan la aplicación. Pruebas específicas con adultos mayores, familias con niños y personas con diferentes niveles de familiaridad tecnológica permitirían identificar ajustes necesarios para hacer la aplicación más inclusiva y accesible a toda la diversidad de visitantes potenciales del parque.

    \item La incorporación de elementos multimedia adicionales como narración de audio, videos cortos de contexto histórico o animaciones que ilustren procesos constructivos de las estructuras mayas podría enriquecer significativamente el valor educativo de la aplicación. Estos elementos deberían diseñarse cuidadosamente para complementar, no competir con, la experiencia de realidad aumentada visual que constituye el elemento central de la aplicación.

    \item La validación arqueológica formal del contenido mediante colaboración con expertos del Ministerio de Cultura y Deportes o investigadores académicos especializados en Kaminaljuyú fortalecería la credibilidad científica de la aplicación. Esta validación podría también identificar oportunidades para enriquecer el contenido con información adicional sobre descubrimientos recientes, interpretaciones actualizadas o contextos históricos más amplios que conecten Kaminaljuyú con otras ciudades mayas de Mesoamérica.
\end{itemize}
