% !TEX root = z-main.tex

Esta investigación siguió una metodología cualitativa basada en el enfoque de diseño centrado en el usuario. Este enfoque buscó que algunas personas que usarían la aplicación participaran en las distintas etapas del proyecto, para asegurar que el resultado final se adaptara bien a sus necesidades, expectativas y capacidades. El objetivo fue crear una aplicación de recorrido con Realidad Aumentada (RA) para el Parque Arqueológico Kaminaljuyú que fuera fácil de usar, llamativa y útil. A continuación, se describen las etapas del proceso:

\section{Investigación de usuarios}

La primera etapa consistió en conocer mejor a los visitantes del parque y entender cómo era actualmente su experiencia. Para ello se utilizaron dos métodos principales:

\begin{itemize}
    \item \textbf{Observaciones en campo}, donde se anotó cómo se comportaban los visitantes, por dónde caminaban, qué lugares llamaban más su atención y cómo interactuaban con el entorno.
    \item \textbf{Entrevistas semiestructuradas}, que se realizaron a distintos tipos de visitantes para conocer qué esperaban de su visita, qué les interesaba, qué dificultades encontraban y qué opinaban del parque.
\end{itemize}

Luego, toda esta información se organizó y se buscaron patrones o ideas comunes que ayudaron a definir mejor el diseño de la aplicación.

\section{Definición de requerimientos}

Con base en los hallazgos anteriores, se establecieron los requerimientos que debía cumplir la aplicación. Estos se dividieron en:

\begin{itemize}
    \item \textbf{Requerimientos funcionales}, es decir, lo que la aplicación debía hacer.
    \item \textbf{Requerimientos de diseño}, que tenían que ver con la experiencia de uso y la apariencia.
\end{itemize}

\section{Diseño de prototipos}

En esta fase se realizaron versiones iniciales del diseño de la aplicación, primero más simples (baja fidelidad) y luego más completas (alta fidelidad), usando herramientas como Figma. Esto incluyó:

\begin{itemize}
    \item La organización del contenido y navegación.
    \item El diseño visual, eligiendo colores, letras y botones que transmitieran sensaciones como calma, interés o claridad.
    \item La simulación de recorridos, para imaginar cómo usaría una persona la app durante su visita y comprobar que todo estuviera bien ordenado.
\end{itemize}

\section{Implementación técnica}

Esta etapa incluyó la programación de la aplicación utilizando tecnologías de RA. Aquí se desarrolló la app usando la API Geoespacial de ARCore, lo que permitió colocar modelos en 3D en lugares reales dentro del parque, como los montículos. También se cuidó que la app funcionara bien en celulares y que cargara rápido.

\section{Pruebas de usabilidad}

Durante y después del desarrollo se realizaron pruebas con usuarios reales para comprobar si la aplicación era clara, útil y agradable. Se llevaron a cabo:

\begin{itemize}
    \item \textbf{Pruebas en el parque}, con personas que usaron versiones preliminares de la app.
    \item \textbf{Observaciones y entrevistas} después del uso para saber qué funcionó y qué se podía mejorar.
    \item \textbf{Revisión de métricas básicas}, como cuánto tiempo usaban la app, si lograban completar los recorridos o si encontraban errores.
\end{itemize}

Los resultados permitieron hacer cambios y mejorar el diseño antes de finalizar el proyecto.

\section{Elaboración de reporte de implementación}

Como cierre, se elaboró un reporte que recopiló:

\begin{itemize}
    \item Todo el proceso de diseño y desarrollo de la app.
    \item Las prácticas que funcionaron bien en cuanto a experiencia de usuario e interfaz.
    \item Consejos para futuras versiones y para aplicar esta solución en otros parques o sitios culturales.
\end{itemize}