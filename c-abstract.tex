% !TEX root = z-main.tex

% Abstract en inglés del trabajo de graduación
% Debe ser una traducción fiel del resumen en español

During the development of this project, a mobile augmented reality (AR) tour application was designed and implemented for the Kaminaljuyú Archaeological Park, aiming to enhance the visitor experience through an intuitive, accessible, and visually appealing interface. In response to the lack of contextual signage and limited emotional connection to the site, a technological solution was developed based on user-centered design (UX) and user interface (UI) principles to enrich the interpretation of the archaeological heritage.

The Android application integrates ARCore Geospatial API for precise positioning of three-dimensional models of archaeological structures over the real environment. The system implements a guided tour through 11 points of interest with navigation via directional indicators, contextual information cards, and a proximity system that dynamically guides the user. By integrating digital elements over the physical environment, the application facilitated historical understanding of the park's mounds and structures, while promoting active visitor engagement.

The methodology involved user research, requirements definition, iterative prototyping, technical development, and usability testing with 17 participants --all focused on delivering a seamless, inclusive, and culturally meaningful experience. Usability results demonstrated high effectiveness, with averages between 4.27 and 4.77 on Likert scales (1-5) for ease of use, 100\% task completion rates for critical functions, and positive evaluation of educational value (4.18/5.0). Specific areas for improvement were identified in the minimap (64.7\% usefulness) and geometric clarity of 3D models (35.3\% positive evaluation).

This approach successfully strengthened the connection between users and the pre-Columbian past and established a replicable model for applying emerging technologies in heritage contexts from a UX/UI design perspective.