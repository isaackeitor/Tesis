% !TEX root = z-main.tex

% Abstract en inglés del trabajo de graduación
% Debe ser una traducción fiel del resumen en español

This project designed and implemented an augmented reality (AR) tour application for the Kaminaljuyú Archaeological Park, aiming to enhance the visitor experience through an intuitive, accessible, and visually appealing interface. In response to the lack of contextual signage and limited emotional connection to the site, a technological solution was developed based on user-centered design (UX) and user interface (UI) principles to enrich the interpretation of the archaeological heritage.

By integrating digital elements over the physical environment, the application facilitated historical understanding of the park's mounds and structures, while promoting active visitor engagement. The methodology involved user research, requirements definition, iterative prototyping, technical development, and usability testing --all focused on delivering a seamless, inclusive, and culturally meaningful experience. This approach successfully strengthened the connection between users and the pre-Columbian past and established a replicable model for applying emerging technologies in heritage contexts from a UX/UI design perspective.