% !TEX root = z-main.tex

% En el resumen debes proporcionar una síntesis clara y concisa de tu trabajo
% Debe incluir: objetivo, metodología, resultados principales y conclusiones
% Límite recomendado: 250-300 palabras

Este proyecto diseñó e implementó una aplicación de recorrido con Realidad Aumentada (RA) orientada al Parque Arqueológico Kaminaljuyú, con el objetivo de mejorar la experiencia del visitante a través de una interfaz intuitiva, accesible y visualmente atractiva. Frente a la falta de señalización contextualizada y la escasa conexión emocional con el sitio, se planteó una solución tecnológica basada en principios de diseño centrado en el usuario (UX) y diseño de interfaz (UI), que permitió enriquecer la interpretación del patrimonio arqueológico.

A través de la integración de elementos digitales superpuestos sobre el entorno físico, la aplicación facilitó la comprensión histórica de los montículos y estructuras del parque, al tiempo que promovió la interacción activa del visitante. La metodología empleada contempló fases de investigación con usuarios reales, definición de requerimientos, prototipado iterativo, desarrollo técnico y pruebas de usabilidad, todo ello enfocado en garantizar una experiencia fluida, inclusiva y culturalmente significativa. Este enfoque logró fortalecer la conexión del usuario con el pasado precolombino, y estableció un modelo replicable para la aplicación de tecnologías emergentes en contextos patrimoniales desde la perspectiva del diseño UX/UI.