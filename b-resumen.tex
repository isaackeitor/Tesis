% !TEX root = z-main.tex

% En el resumen debes proporcionar una síntesis clara y concisa de tu trabajo
% Debe incluir: objetivo, metodología, resultados principales y conclusiones
% Límite recomendado: 250-300 palabras

Durante la realización de este proyecto se diseñó e implementó una aplicación móvil de recorrido con Realidad Aumentada (RA) orientada al Parque Arqueológico Kaminaljuyú, con el objetivo de mejorar la experiencia del visitante a través de una interfaz intuitiva, accesible y visualmente atractiva. Frente a la falta de señalización contextualizada y la escasa conexión emocional con el sitio, se planteó una solución tecnológica basada en principios de diseño centrado en el usuario (UX) y diseño de interfaz (UI), que permitió enriquecer la interpretación del patrimonio arqueológico.

La aplicación desarrollada para Android integra ARCore Geospatial API para el posicionamiento preciso de modelos tridimensionales de estructuras arqueológicas sobre el entorno real. El sistema implementa un recorrido guiado por 11 puntos de interés con navegación mediante indicadores direccionales, fichas informativas contextuales, e un sistema de proximidad que orienta al usuario dinámicamente. A través de la integración de elementos digitales superpuestos sobre el entorno físico, la aplicación facilitó la comprensión histórica de los montículos y estructuras del parque, al tiempo que promovió la interacción activa del visitante.

La metodología empleada contempló fases de investigación con usuarios reales, definición de requerimientos, prototipado iterativo, desarrollo técnico y pruebas de usabilidad con 17 participantes, todo ello enfocado en garantizar una experiencia fluida, inclusiva y culturalmente significativa. Los resultados de usabilidad demostraron alta efectividad, con promedios entre 4.27 y 4.77 en escalas Likert (1-5) para facilidad de uso, tasas de completitud de tareas del 100\% en funciones críticas, y evaluación positiva del valor educativo (4.18/5.0). Se identificaron áreas de mejora específicas en el mapa en miniatura (64.7\% utilidad) y claridad geométrica de modelos 3D (35.3\% evaluación positiva).

Este enfoque logró fortalecer la conexión del usuario con el pasado precolombino y estableció un modelo replicable para la aplicación de tecnologías emergentes en contextos patrimoniales desde la perspectiva del diseño UX/UI.