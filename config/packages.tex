% ============================================================================
% packages.tex - Paquetes centralizados para la tesis
% ============================================================================
% Mantén aquí solo lo necesario para reducir tiempos de compilación.
% Añade paquetes extra solo cuando realmente los uses en el texto.
% ============================================================================

% Codificación y lenguaje
\usepackage[utf8]{inputenc}
\usepackage[T1]{fontenc}
\usepackage[spanish, es-nodecimaldot, es-noquoting]{babel}
\selectlanguage{spanish}

% Matemáticas y símbolos
\usepackage{amsmath, amssymb, amsfonts, amsthm, mathtools}

% Gráficos e imágenes
\usepackage{graphicx}
\usepackage{float}      % para [H]
\usepackage{subcaption}
\usepackage[font=small]{caption}
\usepackage[percent]{overpic}
\usepackage{xfrac}

% Formato del documento
\usepackage[top=1in,left=1.5in,right=1in,bottom=1in]{geometry}
\usepackage{fancyhdr}
\usepackage{emptypage}
\usepackage{hyphenat}
\usepackage{chngcntr}
\usepackage[Sonny]{fncychap}

% Colores
\usepackage{xcolor}
\definecolor{uvg-green}{RGB}{17,71,52}

% Hipervínculos y referencias
\usepackage[hypertexnames=false,hidelinks]{hyperref}
\usepackage{csquotes}

% Bibliografía (cambiar estilo activando \usarAPA en 0-datos_estudiante.tex)
\ifdefined\usarAPA
	\usepackage[backend=biber,style=apa,sorting=nyt]{biblatex}
	\DeclareLanguageMapping{spanish}{spanish-apa}
\else
	\usepackage[backend=biber,style=ieee]{biblatex}
\fi
\addbibresource{bibliografia/referencias.bib}

% Glosario opcional
\ifdefined\CAPglosario
	\usepackage[numberedsection]{glossaries}
	\makeglossaries
	% !TEX root = z-main.tex

\newglossaryentry{api}
{
    name=API,
    description={Application Programming Interface - Interfaz de programación de aplicaciones}
}

\newglossaryentry{framework}
{
    name=framework,
    description={Marco de trabajo que proporciona una estructura estándar para el desarrollo de aplicaciones}
}

\newglossaryentry{backend}
{
    name=backend,
    description={Parte del sistema que se ejecuta en el servidor y no es visible directamente al usuario}
}

\newglossaryentry{frontend}
{
    name=frontend,
    description={Parte del sistema que interactúa directamente con el usuario (interfaz de usuario)}
}
\fi

% Contadores globales (figuras/tablas no reinician por capítulo)
\counterwithout{figure}{chapter}
\counterwithout{table}{chapter}
\counterwithout{equation}{chapter}

% Comandos personalizados
\newcommand{\blankpage}{\newpage\thispagestyle{empty}\mbox{}\newpage}
\newcommand{\defaultparformat}[1]{%
	{\setlength{\parskip}{2ex}%
     \input{#1}}%
}

% Formato de capítulos y secciones UVG (opcional)
\ifdefined\capsecuvg
	\renewcommand\thechapter{\Roman{chapter}}
    \renewcommand\thesection{\Alph{section}}
	\renewcommand\thesubsection{\arabic{subsection}}
    \renewcommand\thesubsubsection{\alph{subsubection}}
\fi

% Ruta de gráficos
\graphicspath{ {figuras/} }
